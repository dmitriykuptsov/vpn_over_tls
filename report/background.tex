\section{Background}
\label{section:background}

There are various solutions which can be used to build VPNs.
One example, is Host Identity Protocols (HIP)~\cite{hip}. 
HIP is a layer 3.5 solution (it is in fact located between transport
and network layers) and was originally designed to 
split the dual role of IP addresses - identifier and locator.
For example, a company called Tempered Networks uses HIP protocol
to build secure networks (for sampling see~\cite{temperednetworks}).

Another solution is Secure Shell protocol (or SSH)~\cite{ssh}. SSH is
application layer protocol which provides an encrypted channel for insecure
networks. SSH was originally designed to provide secure  
remote command-line, login, and command execution. But in fact, 
any network service can be secured with SSH. Moreover, SSH provides
means for creating VPN tunnels between the spatially separated networks.
Unfortunately, SSH connection can be analyzed and blocked (would it be as 
widely spread as for example example TLS protocol, things could be different).

Like SSH, OpenVPN~\cite{openvpn} runs on top of UDP or TCP transport protocols.
We have evidence that in certain countries OpenVPN is successfully blocked 
by governments. Of course, it is harder to detect this protocols because 
the traffic is encapsulated inside TCP/UDP connections and deep packet
inspection solutions are required in order to effectively block such tunnels.

Another widely used layer 3 protocol for building the VPNs is IPSec 
protocol~\cite{ipsec}. In IPSec security association can be established using 
pre-shared keys or using Internet Key Exchange protocols (IKE and IKEv2)~\cite{ike}.
Because IPsec runs directly on top of IP protocol, it can be easily detected without
usage of sophisticated packet inspection solutions.
  